\documentclass[11pt]{article}

\usepackage[a4paper,fancysections,titlepage]{polytechnique}
\usepackage[french]{babel}
\usepackage[T1]{fontenc}
\usepackage{blindtext}
\usepackage[hidelinks]{hyperref}
\usepackage{amsmath,amssymb}
\usepackage{pdfpages}
\usepackage{setspace}

\setlength{\parskip}{1em}
\setstretch{1}

\newcommand{\code}[1]{%
    \mbox{\ttfamily%
        \detokenize{#1}%
    }%
}

\newcommand{\resultat}[1]{%
    \quad \rightsquigarrow \quad #1%
}

% \logo{delos_logo.jpeg}
\author{Nathan Duboisset \and Rom\'eo Nazaret}
\date{\today}
\title{D\'etection de pupilles par vision par ordinateur}
\subtitle{Projet CSC51073 -- 2025}

\begin{document}

\maketitle

\tableofcontents

\newpage
\section{Introduction}
\subsection{Contexte et motivation}
% [TODO: D\'ecrire le contexte du projet, pourquoi la d\'etection de pupilles est importante]

\subsection{Probl\'ematique}
% [TODO: D\'ecrire les d\'efis techniques: robustesse, variations lumineuses, temps r\'eel]

\subsection{Objectifs du projet}
% [TODO: Lister les objectifs sp\'ecifiques du projet]

\newpage
\section{\'Etat de l'art}
\subsection{Algorithme AFIG 2007}
% [TODO: Pr\'esenter l'article de r\'ef\'erence et l'approche morphologique]

\subsection{MediaPipe Face Mesh}
% [TODO: D\'ecrire l'outil MediaPipe et son utilisation pour les rep\`eres faciaux]

\subsection{Autres approches}
% [TODO: Discuter d'autres m\'ethodes existantes (deep learning, approches g\'eom\'etriques, etc.)]

\newpage
\section{M\'ethode impl\'ement\'ee}
\subsection{Architecture globale}
% [TODO: D\'ecrire l'architecture du syst\`eme: pipeline de traitement]

\subsection{D\'etection des rep\`eres faciaux}
% [TODO: Expliquer l'utilisation de MediaPipe pour localiser les yeux]

\subsection{Algorithme de d\'etection pupillaire}
% [TODO: D\'etailler l'algorithme morphologique bas\'e sur Cm et Cc]
\subsubsection{Coefficient morphologique $C_m$}
% [TODO: Expliquer le calcul de Cm]

\subsubsection{Coefficient colorim\'etrique $C_c$}
% [TODO: Expliquer le calcul de Cc]

\subsubsection{Proc\'edure en deux passes}
% [TODO: D\'ecrire pass1 (balayage grossier) et pass2 (raffinement)]

\subsection{Pr\'e-traitement et post-traitement}
% [TODO: D\'ecrire les \'etapes de normalisation, recadrage, filtrage]

\newpage
\section{Impl\'ementation}
\subsection{Organisation du code}
% [TODO: D\'ecrire la structure du projet: python_vision/, src/, data/]

\subsection{Module Python}
% [TODO: D\'etailler les scripts principaux: implementationAlgo2007.py, etc.]

\subsection{Biblioth\`eque C++}
% [TODO: Pr\'esenter EyeToolKit et son r\^ole]

\subsection{Acquisition et journalisation des donn\'ees}
% [TODO: Expliquer le syst\`eme de capture et de logging (csv_log.py, data/)]

\newpage
\section{R\'esultats}
\subsection{Protocole exp\'erimental}
% [TODO: D\'ecrire le protocole de test: donn\'ees utilis\'ees, m\'etriques]

\subsection{Performances quantitatives}
% [TODO: Pr\'esenter les r\'esultats chiffr\'es: pr\'ecision, latence, taux de d\'etection]

\subsection{Analyse qualitative}
% [TODO: Discuter des cas de r\'eussite et d'\'echec, visualisations]

\subsection{Comparaison avec l'\'etat de l'art}
% [TODO: Comparer avec d'autres m\'ethodes si possible]

\newpage
\section{Discussion}
\subsection{Forces de l'approche}
% [TODO: Identifier les points forts de la m\'ethode]

\subsection{Limites et d\'efis}
% [TODO: Discuter des limites: variations inter-individuelles, conditions lumineuses, etc.]

\subsection{Am\'eliorations possibles}
% [TODO: Proposer des pistes d'am\'elioration]

\newpage
\section{Conclusion}
% [TODO: R\'esumer le projet, les r\'esultats et les perspectives]

\newpage
\appendix
\section*{R\'ef\'erences}
\addcontentsline{toc}{section}{R\'ef\'erences}
\begin{thebibliography}{9}
\bibitem{afig2007}
B.~Raynal.
\newblock \emph{Reconnaissance de la pupille par morphologie math\'ematique}.
\newblock Actes de l'AFIG, 2007.
\end{thebibliography}

\end{document}
